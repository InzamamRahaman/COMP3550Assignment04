



\section{Model}

A web application that displays agricultural data to users. An application is a warehouse that monitors the current stock. Both of these applications share an important feature - there is some underlying data model that we want to operate upon in some manner in accord with constraints and  domain logic. 

Abstracting over the underlying data model and its domain logic that is to be used by an application, and providing services to access and manipulate the data in accord with said domain logic is the responsibility of the \textbf{Model}.

In OOD, it is not uncommon for elements of data model or domain to represented as Objects that either encapsulate the data directly, or act as an intermediary between the software and the database system using an ORM.


\section{View}

For our users to view their requested data, our application would need some mechanism of present our data to our users. The \textbf{View} refers to the component or components that facilitate this task. 

Since, there can be more than one way to display the data to the user, a single application may have multiple Views - one for each way of presenting data to our end users.

For example, in our web application, our View(s) would comprise our client side markup - our HTML and CSS. Similarly, for our warehousing application, our View(s) would be the display components of our UI that facilitates the presentation of data to our users.

\section{Controller}

In order to request services from our Model in a manner decoupled from our View, we need some component that acts as a middleman, translating between the two and requesting appropriate services from the Model on the behalf of the end-user utilizing the View. This is the responsibility of the \textbf{Controller}.

Since, different Views may have different service requirments of the Model, there is typically a one-to-one mapping between Views and Controllers. As such, an application can have many Controllers. 

In addition to supplying the View with a mechanism for interacting with the Model, the Controller can also manipulate the View. For example, the Controller can disable a button in response to notifications sent or delivered by the Model. In some implementations of MVC, their is no clear distinction between a View and its Controller. 

\section{Diagramtic Summary}

\begin{center}

\begin{tikzpicture}[node distance=2cm,
 >=latex',
  auto,
  thick
]

\node (c1) [component] {View};

\node (c2) [component, right of=c1, xshift=10cm] {Controller};

\node (c3) [component, below of=c2, yshift=-8cm] {Model};

%\draw [arrow] (c1) -- node[anchor=north] {request operation} (c2);
%\draw [arrow] (c2) -- node[anchor=east] {request services}(c3);
%\draw [arrow] (c3) -- node [anchor=north] {consume services and elicit changes};(c2)
%\draw [arrow] (c2) -- [anchor=west] {Make changes} (c1);
\path[->,shift left=.75ex]
    (c1) edge node {request operation}   (c2)
    (c2) edge node {request service} (c3)
    (c3) edge node {consume service} (c2)
    (c2) edge node {changes} (c1);


\end{tikzpicture}


\end{center}


\section{An example of how it solves the problem}


Let us consider the case of an application that renders a table, a bar chart, and a line graph, each on a separate page, for the user. 


\begin{enumerate}
\item The data that is to be presented to the user is initially stored in some database. Our \textbf{Model} is responsible for retrieving the data in our database

\item As aforementioned, our application has three means of presenting information:
    \begin{itemize}
    \item A table
    \item a bar chart
    \item a line graph
    \end{itemize}
    Each of these means of presenting information to the user, would require a different \textbf{View}, comprising  its own UI components
    
    
\item Our application would also need for each view, a \textbf{Controller} that serves as an intermediary between their respective View and the Model - translating between them and eliciting the responsible updates and responses
\end{enumerate}


Now let us examine our application and our above scheme for MVC to understand how MVC helps make our code modular, resuable, and extensible


\begin{enumerate}
\item Recall that we have multiple View, Controller pairs. Each of the View, Controller pairs would need to request and consume similar services from the underlying data model. By encapsulating and abstracting our data model in our Model, we can simply reuse our Model for use in all our View, Controller pairs, thereby reducing the complexity and size of our code base, making our application easier to extend, and increasing the re-usability of facets of our code base.

\item Many applications are developed by a team of people as opposed to a single person. Consequently, it would a great boon if our software architectures can accommodate a parallel approach to development. MVC can also be helpful in this regard. Since the Model and the Views are decoupled and the Controller is responsible for translating between them the Model and the Views can be developed independently of one another, with the concentration being placed on the services that ought to be provided by the Model.

\item 
\end{enumerate}







