
\section{What MVC ?}

Software decays at a faster rate than most other engineering artifacts - the circumstances under which a piece of software must operate can evolve at a rapid pace and render it obsolete. Consequently, software engineering grew largely out of the need for software that could be easier to maintain, debug, and extend as the environment in which it operated evolved, grew, and adapted to new technological, social, and economic mores. An important notion to arise from this need is that software should be developed such that its components exhibit high cohesion and low coupling. If our components exhibit high coupling, then a modifying some component A will directly affect all components that depend on component A.

However, writing software that is decoupled and cohesive is not a trivial task. In light of this fact, software engineers and software developers need methodologies and software architectures that they can leverage to make their code as easy to maintain and as decoupled as possible  while facilitating responsible code-reuse, flexibility, and low code complexity. One of the post popular software architectures used in trying to achieve this is the Model-View-Controller (henceforth referred to as MVC) software architecture.

Let us now analyze MVC along its three facets - Model, View, and Controller in that order

\section{Model}

A web application that displays agricultural data to users. An application is a warehouse that monitors the current stock. Both of these applications share an important feature - there is some underlying data model that we want to operate upon in some manner in accord with constraints and  domain logic. 

Abstracting over the underlying data model and its domain logic that is to be used by an application, and providing services to access and manipulate the data in accord with said domain logic is the responsibility of the \textbf{Model}.

In OOD, it is not uncommon for elements of data model or domain to represented as Objects that either encapsulate the data directly, or act as an intermediary between the software and the database system using an ORM.


\section{View}

For our users to view their requested data, our application would need some mechanism of present our data to our users. The \textbf{View} refers to the component or components that facilitate this task. 

Since, there can be more than one way to display the data to the user, a single application may have multiple Views - one for each way of presenting data to our end users.

For example, in our web application, our View(s) would comprise our client side markup - our HTML and CSS. Similarly, for our warehousing application, our View(s) would be the display components of our UI that facilitates the presentation of data to our users.

\section{Controller}

In order to request services from our Model in a manner decoupled from our View, we need some component that acts as a middleman, translating between the two and requesting appropriate services from the Model on the behalf of the end-user utilizing the View. This is the responsibility of the \textbf{Controller}.

Since, different Views may have different service requirments of the Model, there is typically a one-to-one mapping between Views and Controllers. As such, an application can have many Controllers. 

In addition to supplying the View with a mechanism for interacting with the Model, the Controller can also manipulate the View. For example, the Controller can disable a button in response to notifications sent or delivered by the Model. In some implementations of MVC, their is no clear distinction between a View and its Controller. 

\section{Diagramtic Summary}

\begin{center}

\begin{tikzpicture}[node distance=2cm,
 >=latex',
  auto,
  thick
]

\node (c1) [component] {View};

\node (c2) [component, right of=c1, xshift=10cm] {Controller};

\node (c3) [component, below of=c2, yshift=-8cm] {Model};

%\draw [arrow] (c1) -- node[anchor=north] {request operation} (c2);
%\draw [arrow] (c2) -- node[anchor=east] {request services}(c3);
%\draw [arrow] (c3) -- node [anchor=north] {consume services and elicit changes};(c2)
%\draw [arrow] (c2) -- [anchor=west] {Make changes} (c1);
\path[->,shift left=.75ex]
    (c1) edge node {request operation}   (c2)
    (c2) edge node {request service} (c3)
    (c3) edge node {consume service} (c2)
    (c2) edge node {changes} (c1);


\end{tikzpicture}


\end{center}





