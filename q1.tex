
\section{What MVC ?}

Software decays at a faster rate than most other engineering artifacts - the circumstances under which a piece of software must operate can evolve at a rapid pace and render it obsolete. Consequently, software engineering grew largely out of the need for software that could be easier to maintain, debug, and extend as the environment in which it operated evolved, grew, and adapted to new technological, social, and economic mores. An important notion to arise from this need is that software should be developed such that its components exhibit high cohesion and low coupling. If our components exhibit high coupling, then a modifying some component A will directly affect all components that depend on component A.

However, writing software that is decoupled and cohesive is not a trivial task. In light of this fact, software engineers and software developers need methodologies and software architectures that they can leverage to make their code as easy to maintain and as decoupled as possible  while facilitating responsible code-reuse, flexibility, and low code complexity. One of the post popular software architectures used in trying to achieve this is the Model-View-Controller (henceforth referred to as MVC) software architecture.

Let us now analyze MVC along its three facets - Model, View, and Controller in that order

\section{Model}

A web application that displays agricultural data to users. An application is a warehouse that monitors the current stock. Both of these applications share an important feature - there is some underlying data model that we want to operate upon in some manner. 

Abstracting over the underlying data model that is to be used by an application is the responsibility of the \textbf{Model}

In OOD, it is not uncommon for elements of data model or domain to represented as Objects that either encapsulate the data directly, or act as an intermediary between the software and the database system using an ORM
\section{View}

Our two example applications above also share other important features - they both need some means of allowing users to view data and to request operations of the data (such as inserting records, finding averages, ect...). 

This is activity of presenting data to the user and facilitating operations on the data is the responsibility of the \textbf{View}

\section{Controller}

Having the 