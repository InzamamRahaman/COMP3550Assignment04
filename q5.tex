\section{The future of MVC in software development}

The lion's share of software development takes place across the web and it is likely to increase in the future. This is understandable
as the internet allows for applications to reach a large audience. As time progressed, web developers looked for ways to structure their
applications by turning to software patterns. MVC gained popularity but there are some concerns with how it is treated. A misunderstanding of
MVC has resulted in situations where developers think of data as being handled one way, the view handled another way and everything else being labelled
under `controller`.\par


Angular JS in particular was developed in order to address the root of the issue, that HTML was not defined for dynamic views.\par
\say{AngularJS is a structural framework for dynamic web apps. It lets you use HTML as your template language and lets you extend HTML's
syntax to express your application's components clearly and succinctly.
Angular's data binding and dependency injection eliminate much of the code you would otherwise have to write.
And it all happens within the browser, making it an ideal partner with any server technology.\par

Angular is what HTML would have been had it been designed for applications. HTML is a great declarative language for static documents.
It does not contain much in the way of creating applications, and as a result building web applications is
an exercise in what do I have to do to trick the browser into doing what I want?
}\par

Facebook has chosen to discontinue the use of MVC. Their concern is that MVC does not not scale up for their needs and they have chosen to
develop a different software pattern: Flux. http://facebook.github.io/flux/docs/overview.html. Their reasoning is that for a large codebase, implementing
and managing MVC becomes complicated. System complexity grew exponentially with each feature addition which lead to the code being \say{fragile and upredictable.}
Flux is designed to structure the code in a way that increases the predictability of what action will occur. It promotes single directional data flow
through an application. \par

\say{The Store contains all the application’s data and the Dispatcher replaces the initial Controller, deciding how the Store is to be
updated when an Action is triggered. The View is also updated when the Store changes, optionally generating an Action to be processed
by the Dispatcher. This ensures a unidirectional flow of data between a system’s components. A system with multiple Stores or Views can be
seen as having only one Store and one View since the data is flowing only one way and the different Stores and Views do not directly affect each other.}\par

By using this approach, the Data Layer is allowed to complete the update of the view before any other action is triggered and the Dispatcher can
reject actions if it is currently processing a previous action. This results in cleaner code that is easier to debug by future developers of a system.
