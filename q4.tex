Although MVC is an excellent means of separating presentation and interaction from system data it
does have its disadvantages. \par

\textbf{Increased Complexity.} When the blueprint for an application is being constructed it is important to consider its purpose
and scale. The reason for this is that adhering to the Model-View-Controller
structure is that it is not always the most effective method to build a UI based application. If the application being considered is
to be simple in nature then the use of separate model, view and controller components increases the complexity of development without
much gain. \par

\textbf{Close connection between view and controller.} While the use of the MVC structure mandates that the
view and the controller be separate components, they still need to closely related which places a limit on the individual
reuse of said components. A view would be need to be used with its controller and vice version. The exception to this would be
views that do not trigger updates (read only) and thus they share a controller that ignores all input.\par

\textbf{Close coupling of controllers and views to a given model.} In an MVC structure, the views and the controller components
make direct calls to the model. The implication here is that any changes to the model would likely break the code utilized by the view and
the controller. If the system in question uses multiple views and controllers then the is magnified. \par

\textbf{Inefficient data access from the view.} Depending on how the model is implemented, it may be necessary for a view to
make multiple calls to obtain all of the data required for display. From this, one can see that the view may unnecessarily request unchanged
data from the model thus weakening the performance of the application if the updates occur frequently. The use of data caches within the view can
be used to counter this and improve responsiveness.\par

\textbf{Changes must be made to the view and controller when porting the application.} Dependencies that are related to the user interface module
of a given application are encapsulated within the view and the controller. The components also contain code that is independent of a specific platform.
Thus, porting an MVC application requires the developers to separate the platform dependent code from the platform agnostic code before rewriting
the application for a different platform. In light of this, developers can go a step further and encapsulate platform dependencies as required.\par
